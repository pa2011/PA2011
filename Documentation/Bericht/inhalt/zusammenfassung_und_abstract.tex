\section*{Zusammenfassung}
In Zusammenarbeit mit der ETH-Zürich wurde, für eine bestehende Hardwareumgebung, ein Fahrsimulator entwickelt. Die ETH Zürich forscht im Rahmen verschiedener Themen im Gebiet des Strassenverkehrs. Und benötig für diese Zwecke einen Fahrsimulator, in dem möglichst realistische Verkehrssituationen nachgestellt werden können. \\
Die Hardwareumgebung besteht aus einem Cockpit, das mit einem Steuerrad, Gas- , Brems- und Kupplungspedal bestückt ist, einem Projektor und einem Computer. \\
Unter Verwendung einer Open Source Graphik Engine wurde eine virtuelle Welt geschaffen. Durch Eingaben im Cockpit kann das Fahrzeug durch die virtuellen Welt navigiert werden. Ein besonderes Augenmerk wurde auf eine kurze Reaktionszeit zwischen Ein- und Ausgabe sowie ein möglich intuitives Fahrverhalten des Fahrzeuges gelegt. Im Simulator stehen zwei Ansichten zur Verfügung. Eine Ansicht ist diejenige eines Fahrers durch die Windschutzscheibe des Fahrzeugs und die andere ist eine Drittperson Ansicht bei der die Kamera über dem Auto schwebt. Es stehen zudem zwei verschiedene virtuelle Welten zur Verfügung. Die Stadtumgebung beinhaltet mehrere Strassen die sich kreuzen. An Kreuzungen herrscht eine übliche Verkehrsführung, gekennzeichnet durch Asphaltsymbole und Strassentafeln. Die Strasse in der Berglandschaft beschreibt einen Rundkurs, der sich abwechslungsweise in Tunnels im den Berg oder an der Oberfläche befindet.\\
Alle aktuellen Eigenschaften und Eingaben die im Cockpit gemacht werden, werden für eine Auswertung gespeichert. 
\newpage
\section*{Abstract}
In Zusammenarbeit mit der ETH-Zürich wurde, für eine bestehende Hardwareumgebung, ein Fahrsimulator entwickelt. Die ETH Zürich forscht im Rahmen verschiedener Themen im Gebiet des Strassenverkehrs. Und benötig für diese Zwecke einen Fahrsimulator, in dem möglichst realistische Verkehrssituationen nachgestellt werden können. \\
Die Hardwareumgebung besteht aus einem Cockpit, das mit einem Steuerrad, Gas- , Brems- und Kupplungspedal bestückt ist, einem Projektor und einem Computer. \\
Unter Verwendung einer Open Source Graphik Engine wurde eine virtuelle Welt geschaffen. Durch Eingaben im Cockpit kann das Fahrzeug durch die virtuellen Welt navigiert werden. Ein besonderes Augenmerk wurde auf eine kurze Reaktionszeit zwischen Ein- und Ausgabe sowie ein möglich intuitives Fahrverhalten des Fahrzeuges gelegt. Im Simulator stehen zwei Ansichten zur Verfügung. Eine Ansicht ist diejenige eines Fahrers durch die Windschutzscheibe des Fahrzeugs und die andere ist eine Drittperson Ansicht bei der die Kamera über dem Auto schwebt. Es stehen zudem zwei verschiedene virtuelle Welten zur Verfügung. Die Stadtumgebung beinhaltet mehrere Strassen die sich kreuzen. An Kreuzungen herrscht eine übliche Verkehrsführung, gekennzeichnet durch Asphaltsymbole und Strassentafeln. Die Strasse in der Berglandschaft beschreibt einen Rundkurs, der sich abwechslungsweise in Tunnels im den Berg oder an der Oberfläche befindet.\\
Alle aktuellen Eigenschaften und Eingaben die im Cockpit gemacht werden, werden für eine Auswertung gespeichert. 

