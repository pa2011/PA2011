\section*{Zusammenfassung}
Die ETH Zürich betreibt diverse Forschungsprojekte im Bereich des Strassenverkehrs. Für diese Zwecke benötigt sie einen Fahrsimulator, in dem Verkehrssituationen möglichst realistisch nachgestellt werden können. Das Ziel dieses Projektes ist die Erstellung einer Fahrsimulationssoftware für den bestehenden Hardwareaufbau. Um die neue Software in das existierende Framework einzugliedern, ist eine Anbindung der Hardware über LabVIEW zwingend notwendig.\\
Unter Verwendung von OGRE, einer Open Source Grafik Engine wurde ein funktionsfähiger Fahrsimulator entwickelt, der dem Benutzer die Möglichkeit gibt ein Fahrzeug durch eine virtuelle Stadt oder eine Berglandschaft mit Tunnels zu navigieren. Dabei wird die Perspektive so gewählt, dass der Benutzer die Illusion erhält, tatsächlich im Fahrzeug zu sitzen.
Um eine möglichst hohe Qualität der Forschungsergebnisse zu gewährleisten, wurde ein besonderes Augenmerk auf eine möglichst kurze Reaktionszeit des Systems auf Eingaben sowie ein intuitives und realistisches Fahrverhalten gelegt.
Sämtliche relevanten Daten werden während der Simulation aufgezeichnet und stehen für weitere Auswertungen zur Verfügung.
\newpage
\thispagestyle{empty}
\hspace{1cm}
\newpage
\section*{Abstract}
In cooperation with ETH Zurich a software for a driving simulator has been developed to go with an existing setup of hardware equipment. ETH is doing research on various subjects and therefore needs a driving simulator which offers the possiblility to simulate certain traffic situations as realistically as possible.\\
Using an open source graphics engine a virtual world has been created. Using the input devices in the simulator a vehicle can be controlled within that world. The focus was on minimizing the time delay between the system's input and output and on creating a driving experience as intuitive as possible. The simulator offers two points of view. A first person view which gives the driver the experience of actually sitting inside the vehicle and a third person view with a chase camera positioned outside the car. Furthermore the user can choose between two virtual worlds. A city world containing a road system with traffic signs following the Swiss traffic rules. The second world represents circuit in a mountain region with tunnels.\\
All properties and system inputs are logged in a file for later analysis.
\newpage
\thispagestyle{empty}
\hspace{1cm}
\newpage