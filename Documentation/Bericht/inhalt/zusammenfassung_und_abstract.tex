\section*{Zusammenfassung}
Die ETH Zürich betreibt diverse Forschungsprojekte im Bereich des Strassenverkehrs. Für diese Zwecke benötigt sie einen Fahrsimulator, in dem Verkehrssituationen möglichst realistisch nachgestellt werden können. Das Ziel dieses Projektes ist die Erstellung einer Fahrsimulationssoftware für den bestehenden Hardwareaufbau. Um die neue Software in das existierende Framework einzugliedern, ist eine Anbindung der Hardware über LabVIEW zwingend notwendig.\\
Unter Verwendung von OGRE, einer Open Source Grafik Engine wurde ein funktionsfähiger Fahrsimulator entwickelt, der dem Benutzer die Möglichkeit gibt, ein Fahrzeug durch eine virtuelle Stadt oder eine Berglandschaft mit Tunnels zu navigieren. Dabei wird die Perspektive so gewählt, dass der Benutzer die Illusion erhält, tatsächlich im Fahrzeug zu sitzen.
Um eine möglichst hohe Qualität der Forschungsergebnisse zu gewährleisten, wurde ein besonderes Augenmerk auf eine möglichst kurze Reaktionszeit des Systems auf Eingaben sowie ein intuitives und realistisches Fahrverhalten gelegt.
Sämtliche relevanten Daten werden während der Simulation aufgezeichnet und stehen für weitere Auswertungen zur Verfügung.
\newpage
\thispagestyle{empty}
\hspace{1cm}
\newpage
\section*{Abstract}
The ETH Zurich promotes various  projects in the field of road traffic. For this purpose a driving simulator that recreates traffic situations in the most realistic way possible is required. This projects was conducted in cooperation with the ETH, department of Innovation and Technology Management and aimed at the development of a driving simulation software for the existing hardware installation. In order to integrate the new software in the present framework, it necessarily had to be linked to the hardware via LabView. \\
OGRE, an open source grafic engine, was used to develop an operational driving simulator which enabels the user to navigate in a virtual city or in a montainous landscape including tunnels. The perspective deliberately gives the user the illusion to be sitting behind the driving wheel of a car. To ensure a high quality of the results, great importance was attached to maximally reduce the reaction time of the system executing tasks and to provide a intuitive and realistic driving behaviour. All the relevant data was continuously recorded during the simulation and therefore can be evaluated at a later date.
\newpage
\thispagestyle{empty}
\hspace{1cm}
\newpage