\section{Resultate und Tests}
\subsection{Erreichtes}
Die Schnittstelle zwischen dem Cockpit und dem Fahrsimulator wurde vollständig in LabVIEW realisiert. Um diese zu testen wurde ein Videoplayer implementiert. Der Videoplayer reagiert auf Eingaben im Cockpit. Beim Drücken des Gaspedals wird das Video schneller, bis maximal doppelt so schnell, abgespielt. Bei einem Druck auf das Bremspedal wird das Video langsamer abgespielt. Liegt die Geschwindigkeit unter 15 km/h, wird das Video pausiert.\\
Im Fahrsimulator kann das Fahrzeug durch Eingaben im Cockpit bewegt werden. Es stehen zwei unterschiedliche Szenen zur Verfügung. Eine Szene stellt eine Stadt dar, in welcher mehrere Strassen, die sich kreuzen, vorhanden sind. Die Kreuzungen sind durch Asphaltsymbole und Verkehrsschilder gekennzeichnet. Um die Stadtszene lebendiger zu machen, sind verschiedene Gebäude und Bäume eingefügt worden. Zudem wurde das ETH Zürich Gebäude in die Szene eingefügt. Dies zeigt exemplarisch, wie einfach Objekte aus dem \gls{google-3d-warehouse} eingefügt werden können. Eine zweite Szene ist ein Rundkurs in einer Berglandschaft. Auf diesem Rundkurs gibt es mehrere Tunnels, die duch die Berglandschaft führen.
Die Daten beider Systeme, die des Videoplayers sowie die des Fahrsimulators, werden erfolgreich an das LabVIEW-Programm zurückgeschikt und in ein Log-File geschrieben. Diese Daten können in eine spätere Auswertung mit einbezogen werden.\\
Somit wurden alle Anforderungen an das Projekt erfüllt. Das Resultat betrachten wir als zufriedenstellend. 
\subsection{Zeitliches Verhalten des Systems}
Es wurden für den Videplayer und den Fahrsimulator Zeitmessungen durchgeführt. Bei beiden Tests war das System auf einer einzigen Maschine installiert. Das Einlesen des Cockpits wurde physisch nicht vom System getrennt. Es gibt also keine Verzögerung durch das Netzwerk. Trotz dieser Tatsache war das Resultat überraschend, denn die Verzögerung liegt im Normalfall unter einer Millisekunde.
\subsection{Testfälle}
Es ist im Allgemeinen schwierig eine grafische Applikation wie ein Fahrsimulator durch automatische Tests zu überprüfen. Deshalb wurden die meisten Tests manuell durchgeführt. Dies geschah jeweils mindestens jeden zweiten Montag Nachmittag während vier bis fünf Stunden. In dieser Zeit wurden alle Neuerungen, die während der Woche entwickelt wurden, auf dem Fahrsimulator installiert und eingehend getestet. \\
Häufige Fehlerursachen waren versäumte Initialisierungen von Variabeln, welche auf verschiedenen Betriebssystemen unterschiedlich gehandhabt werden. Da der Fahrsimulator auf unterschiedlichen Betriebssystemen getestet wurde, konnten diese Fehlerquellen weitgehend eliminiert werden. \\
Wir, das Entwicklerteam, sind durch das häufige Testen zur Überzeugung gekommen, dass die Software den Anforderungen an Stabilität und Zuverlässigkeit gerecht wird.