\section{Resultate und Tests}
\subsection{Erreichtes}
Die Schnittstelle zwischen dem Cockpit und dem Fahrsimulator wurde vollständig und in LabVIEW realisiert. Um die Schnittstelle zu testen wurde ein Videoplayer implementiert. Der Videoplayer reagiert auf Eingaben im Cockpit wie gewünscht. Beim drücken des Gaspedals wird das Video schneller bis es maximal doppelt so schnell abgespielt wird. Bei einem Druck auf das Bremspedal wird das Video langsmer abgespielt. Liegt die Geschwindigkeit unter 15 km/h, wird das Video pausiert. Im Fahrsimulator kann das Fahrzeug durch Eingaben im Cockpit bewegt werden. Es stehen zwei unterschiedliche Szenen zur Verfügung. Eine Szene stellt eine Stadt dar. In dieser sind mehrer Strassen die sich kreuzen Vorhanden. Die Kreuzungen sind durch Asphaltsymbole und Verkehrsschilder gekennzeichnet. Um die Stadtszene lebendiger zu machen sind verschiedene Gebäude und Bäume eingefügt worden. Zudem wurde das ETH Zürich Gebäude in der Szene eingefügt. Dies ist ein Beispiel dafür wie einfach Objekte aus dem google 3D Warehouse eingefügt werden können. Eine zweite Szene ist ein Rundkurs in einer Berglandschaft. Auf diesem Rundkurs gibt es mehrere Tunnels die duch die Berglandschaft führen. 
Die Daten beider Systeme, die des Videoplayers sowie die des Fahrsimulators, werden an das LabVIEW-Programm zurückgeschikt und in ein Log-File geschrieben. Dies ermöglicht eine Auswertung.\\
Somit wurden alle Anforderungen an das Projekt erfüllt. Das Resultat ist sehr zufriedenstellend. 
\subsection{Zeitliches Verhalten des Systems}
Es wurden für den Videplayer und den Fahrsimulator Zeitmessungen durchgeführt. Bei den Tests war das System ausschliesslich auf einer Maschiene installiert. Das Einlesen des Cockpits wurde physisch nicht vom System getrennt. Es gibt also keine Verzögerung durch das Netzwerk. Troz dieser Tatsache war das Resultat überaschend, denn die Verzögerung liegt nur zwischen 0 bis maximal einer Milisekunde. Dementsprechend lliegt auch die Verzögerung zwischen Eingaben im Cockpit und dem ragieren des Fahrismulators unter einer halben Milisekunde. Dieses gute Resultat ist sehr zufriedenstellend. 
\subsection{Testfälle}
Es ist im allgemeinen schwierig bis unmöglich einen Fahrsimulator durch automatische Tests zu teste. Deshalb wurden die meisten Tests manuell durchgeführt. Dies geschah jeweils mindestens jeden zweiten Montag Nachmittag während 4-5 Stunden. In dieser Zeit wurden alle Neuerungen die während der Woche entiwckelt wurden, auf den Fahrsimulator geladen und eingehend getestet. \\
Häufige Fehlerursachen waren vergessene Initialisierungen von Variabeln die auf verschiedenen Betriebssystemen unterschiedlich behandelt werden. Da der Fahrsimulator auf unterschiedlichen Betriebssystemen getestet wurde konnnte diese Fehlerquelle weitgehend eliminiert werden. \\
Wir, das Entwicklerteam, ist durch das viele Testen zur Überzeugung gekommen, dass die Software nur noch wenig Fehlerpotential aufweist. Das System wird als robust und stabil betrachtet. 