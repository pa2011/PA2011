\section{Resultate und Tests}
\subsection{Erreichtes}
Die Schnittstelle zwischen dem Cockpit und dem Fahrsimulator wurde vollständig und in LabVIEW realisiert. Um die Schnittstelle zu testen wurde ein Videoplayer implementiert. Der Videoplayer reagiert auf Eingaben im Cockpit wie gewünscht. Beim Drücken des Gaspedals wird das Video schneller bis es maximal doppelt so schnell abgespielt wird. Bei einem Druck auf das Bremspedal wird das Video langsmer abgespielt. Liegt die Geschwindigkeit unter 15 km/h, wird das Video pausiert. Im Fahrsimulator kann das Fahrzeug durch Eingaben im Cockpit bewegt werden. Es stehen zwei unterschiedliche Szenen zur Verfügung. Eine Szene stellt eine Stadt dar. In dieser sind mehrere Strassen, die sich kreuzen vorhanden. Die Kreuzungen sind durch Asphaltsymbole und Verkehrsschilder gekennzeichnet. Um die Stadtszene lebendiger zu machen, sind verschiedene Gebäude und Bäume eingefügt worden. Zudem wurde das ETH Zürich Gebäude in der Szene eingefügt. Dies zeigt exemplarisch, wie einfach Objekte aus dem google 3D Warehouse eingefügt werden können. Eine zweite Szene ist ein Rundkurs in einer Berglandschaft. Auf diesem Rundkurs gibt es mehrere Tunnels, die duch die Berglandschaft führen. 
Die Daten beider Systeme, die des Videoplayers sowie die des Fahrsimulators, werden erfolgreich an das LabVIEW-Programm zurückgeschikt und in ein Log-File geschrieben. Diese Daten können in eine spätere Auswertung mit einbezogen werden.\\
Somit wurden alle Anforderungen an das Projekt erfüllt. Das Resultat ist sehr zufriedenstellend. 
\subsection{Zeitliches Verhalten des Systems}
Es wurden für den Videplayer und den Fahrsimulator Zeitmessungen durchgeführt. Bei den Tests war das System ausschliesslich auf einer Maschine installiert. Das Einlesen des Cockpits wurde physisch nicht vom System getrennt. Es gibt also keine Verzögerung durch das Netzwerk. Trotz dieser Tatsache war das Resultat überraschend, denn die Verzögerung liegt bei lediglich null bis maximal einer Milisekunde. Dementsprechend ist auch die Verzögerung zwischen Eingaben im Cockpit und dem Reagieren des Fahrismulators unter einer halben Milisekunde. Dieses Resultat ist sehr zufriedenstellend. 
\subsection{Testfälle}
Es ist im Allgemeinen schwierig, bis unmöglich einen, Fahrsimulator durch automatische Tests zu testen. Deshalb wurden die meisten Tests manuell durchgeführt. Dies geschah jeweils mindestens jeden zweiten Montag Nachmittag während vier bis fünf Stunden. In dieser Zeit wurden alle Neuerungen, die während der Woche entwickelt wurden, auf den Fahrsimulator geladen und eingehend getestet. \\
Häufige Fehlerursachen waren versäumte Initialisierungen von Variabeln, welche auf verschiedenen Betriebssystemen unterschiedlich gehandhabt werden. Da der Fahrsimulator auf unterschiedlichen Betriebssystemen getestet wurde, konnten diese Fehlerquellen weitgehend eliminiert werden. \\
Wir, das Entwicklerteam, sind durch das viele Testen zur Überzeugung gekommen, dass die Software nur noch wenig Fehlerpotential aufweist. Das System wird als robust und stabil betrachtet. 