\section{Nächste Schritte}
\subsection{Offene Punkte}
Trotz einem gutem Endergebniss der Arbeit, konnten einige Punkte aus Zeitgründen nicht realisiert werden. Einer dieser Punkte ist die implementierung von anderen Verkehrsteilnemern wie zum Beispiel ander Autos, Passanten oder Velofahrer. Zusätzlich hätte dann auch eine Kolisionserkennung implementiert werden müssen, um Zusammenstösse mit ruhenden oder bewegten Objekten zu erkennen. Um der Illusion der realen Welt etwas näher zu kommen, wäre eine implementation eines eigenen \glspl{shader} von grossem Vorteil. Um das Starten des Fahrsimulators zu erleichtern wäre ein \gls{gui} eine gute Lösung gewesen. 
\subsection{Zusätzliche Funktionen}
Um den Fahrsimulator unabhänging von den Szenen zu machen, die manuell erstellt werden müssen, könnten die Szenen aus Informationen von google maps automatisch generiert werden. Auf google maps sind die Informationen über Lage, Art und Richtung der Strasse verfügbar. Ein Algorithmus könnte mit diesen Informationen Strassen für jedes beliebige Gebiet generieren. \\
Eine weitet Zusätzliche Funktion wäre ein erweitertes \gls{gui}, mit dem manipulationen am Fahrzeug vorgenommen werden könnten. Zum Beispiel könnte die Leistung des Fahrzeuges oder die Helligkeit der Szene verändert werden. 
\subsection{Ausblick auf Bachelor Arbeit 2012}
Da die Zusammenarbeit mit der ETH sehr erfolgreich war und das Projekt sehr Zufriedenstellend verlief, wurde von den zuständigen Dozenten die Weiterführung der Projektarbeit als Bachelor Arbeit beschlossen. Wir begrüssen diese Entscheidung sehr und freuen uns dieses Thema noch weiter vertiefen zu dürfen. Wir freuen uns auf eine weiterhin gute und konstruktive Zusammenarbeit. 