\section{Nächste Schritte}
\subsection{Offene Punkte}
Trotz einem gutem Endergebnis der Arbeit, konnten einige Punkte aus Zeitgründen nicht realisiert werden. Einer dieser Punkte ist die Implementierung von interaktivem Verkehr mit anderen Autos, Passanten oder Velofahrern. Zusätzlich hätte dann auch eine Kollisionserkennung implementiert werden müssen, um Zusammenstösse mit ruhenden oder bewegten Objekten zu erkennen. Um der Illusion der realen Welt etwas näher zu kommen, wäre eine Implementation eines eigenen \glspl{shader} von grossem Vorteil. Um das Starten des Fahrsimulators zu erleichtern wäre ein \gls{gui} eine gute Lösung.
\subsection{Zusätzliche Funktionen}
Um den Fahrsimulator von manuell erstellten Szenen unabhänging zu machen, könnten die Szenen aus Informationen von Google Maps automatisch generiert werden. Auf Google Maps sind die Informationen über Lage, Art und Richtung der Strasse verfügbar. Ein Algorithmus könnte mit diesen Informationen Strassen für jedes beliebige Gebiet generieren.\\
Eine zusätzliche Funktion wäre ein erweitertes \gls{gui}, mit dem Manipulationen am Fahrzeug oder der Szene vorgenommen werden können. Zum Beispiel könnte die Leistung des Fahrzeuges Verkehrsdichte verändert werden.\\
Eine Möglichkeit, die Illusion des Fahrens zu verbessern, wäre der Einsatz von Motorengeräuschen und anderen Soundeffekten, wie zum Beispiel quitschenden Reifen bei starkem Bremsen.
\subsection{Ausblick auf Bachelor Arbeit 2012}
Da die Zusammenarbeit mit der ETH sehr erfolgreich war und das Projekt zufriedenstellend verlief, wurde von den zuständigen Dozenten die Weiterführung der Projektarbeit als Bachelor Arbeit beschlossen. Wir begrüssen diese Entscheidung sehr und freuen uns, ein interessantes Projekt weiterführen zu dürfen.

