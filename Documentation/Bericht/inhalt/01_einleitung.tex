\section{Einleitung}
\subsection{Ausgangslage}
Im Gebiet der Fahrsimulatoren gibt es bereits eine vielzahl von verschiedenen Lösungen. Einige Lösungen bestehen aus Filmmaterial das abgespielt wird und der Fahrer muss auf die Bremse drücken sobald ein bestimmtes Ereigniss eintritt. Andere Fahrsimulationen bringen bereits eine virtuelle Welt mit in der man sich mehr oder weniger Frei bewegen bzw. Frei fahren kann. Jedoch sind bei den meisten von diesen Fahrsimulatoren bereits feste Szenarien implementiert die man nicht ändern kann. 

Die Grenzen liegen vorallem in der Leistungsfähigkeit des Rechners auf dem die Simulation installiert werden soll. Reaktionszeit zwischen den Systemen

Das Projekt wird in Zusammenarbeit mit der ETH-Zürich durchgeführt. 


\subsection{Aufgabenstellung}
\subsubsection{Formulierung}
Das Ziel der Arbeit ist einen Fahrsimulator für die bestehende Simulatinsumgebung zu erstellen. Die Simulationsumgebung besteht aus einem Fahrercockpit, einer Leinwand, einem Beamer und einem Computerterminal um die simulation zu steuern. Das Fahrercockpit enthält ein Steuerrad, 3 Pedalen für Gas, Bremsen und Kupplung, einen Schaltknüppel und einen Autositz mit Sicherheitsgurt.

Die Fahrsimulation sollte dem Benutzer die Illusion des Autofahren möglichts realistisch vermitteln. Die soll erreicht werden durch einen Einsatz von Karteninfomationen von Google Maps oder Google Street View. 

\subsubsection{Aufteilung der Arbeit}
Wir teilen die Arbeit in zwei Teile auf. Im ersten Teil werden wir den Fokus vorallem auf die korrekte Ansteuerung des Cockpits legen. Um dies zu testen setzten wir uns das Ziel, dass wir ein Video abspielen und mit den Pedalen Gas und Bremsen die Geschwindigkeit kontrollieren können. In einem zweiten Teil werden wir dann die Fahrsimulation implementieren und installieren.