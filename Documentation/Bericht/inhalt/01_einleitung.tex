\section{Einleitung}
\subsection{Ausgangslage}

Im Gebiet der Fahrsimulatoren gibt es bereits eine Vielzahl von verschiedenen Lösungen. Einige davon bestehen aus Filmmaterial, das abgespielt wird und der Fahrer muss auf die Bremse drücken sobald ein bestimmtes Ereigniss eintritt. Andere Fahrsimulationen bringen bereits eine virtuelle Welt mit, in der man sich mehr oder weniger frei bewegen bzw. frei fahren kann. Jedoch sind bei den meisten von diesen Fahrsimulatoren bereits feste Szenarien implementiert die  nicht geändert werden können.

Die Grenzen liegen vor allem in der Leistungsfähigkeit des Rechners auf dem die Simulation installiert werden soll.

Das Projekt wird in Zusammenarbeit mit der ETH Zürich durchgeführt. Es ist bereits eine LabView Schnittstelle für das Steuerrad vorhanden.


\subsection{Aufgabenstellung}
\subsubsection{Formulierung}

Das Ziel der Arbeit besteht darin, einen Fahrsimulator für die bestehende Simulatinsumgebung zu erstellen. Diese besteht aus einem Fahrercockpit, einer Leinwand, einem Projektor und einem Computerterminal, von dem aus die Simulation gesteuert werden kann. Das Fahrercockpit enthält ein Steuerrad, drei Pedalen, einen Schaltknüppel und einen Autositz mit Sicherheitsgurt.

Die Fahrsimulation sollte dem Benutzer die Illusion des Autofahren möglichts realistisch vermitteln. Die soll durch einen Einsatz von Karteninfomationen von Google Maps oder Google Street View unterstützt werden. 

Zudem sollen alle Betriebszustände und Benutzereingaben registriert und aufgezeichnet werden um eine genaue Analyse zu ermöglichen. 


\subsubsection{Aufteilung der Arbeit}
Wir teilten die Arbeit im Wesentlichen in zwei Teile auf. Im ersten Teil legten wir den Fokus auf die korrekte Ansteuerung des Cockpits. Um dies zu testen, setzten wir uns das Ziel, ein Video abzuspielen und mit Gas- und Bremspedal die Geschwindigkeit kontrollieren zu können. Im zweiten Teil folgte die Implementation und Installation der Fahrsimulation.

\subsection{Zeitplan und Arbeitsteilung}