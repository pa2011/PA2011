\section{Journal}
\minisec{26.9.2011}
Driving Simulator with steering wheel and pedals connected to the PC. There is a LabVIEW Interface which read the input from the Cockpit. Rudy would like to have in a first Step a driving movie which is controlled only by the pedals. --> modified Frame Rate \\
After this she'd like to have a driving Simulator with focus on tunnel entrance an exits.  
\minisec{03.10.2011}
The driving simulator could be a complete fantasy environment or it could be a virtual copy of a real environment. \\
Google Street-View is no possibility to generate a 3D reality because there are to many distraction in it. For example the other cars, people and traffic jam. The street view does not clearly distinguish between street and surfaces of other objects. Also sometimes there are only pictures on the wrong side of the street.  Another argument not to use Google Street-View is there are too larg distances between the pictures, so the rendering is not fluently.\\
We would like to use the information from google earth to build a city like Zurich. We use the street location information to build our own streets and we try to render already existing 3D-models from google eart into our virtual world. \\
There is a difficulty about google earth. there are no information about the height of the streets. (Tunnel, Bridge, etc.) A Possibility could be to implement it manually or to ignore these things. If we implement it manually we should define a area were such things are made. \\
Use UDP-Socket to extract data out of LabVIEW into our program. \\
external Program to control the video --> Program use UDP-Socket\\
In a first step we use LabVIEW to control an external application which plays a video with a configurable fram rate. 
\minisec{10.10.2011}
The Journal and the timetabele had set up and are ok. There has to be an english version of the timetable. \\
We have agreed that we would create  our own 3-D World and expand them with some buildings from google ware house. These are already finished 3-D Models from different buildings. We have to build the streets by ourselves because the streets in google earth are not as good as we wish they should be. We could also create a tunnel in our own 3-D world, what in a rendering from google street view is very difficult and also then the result would be not satisfying. \\
We have showed the video we controlled with LabVIEW and present the ogre frame work we would like to use. 
\minisec{17.10.2011}
We have to calculate the delay time of the user interaction. A Timestamp would be very helpful to study the delays. That's important for the further head- and eye-tracker studies. \\
The Video is now controlled by the pedals, played in mplayer.  There is a batch file to start different videos. 
We switched our repository to github because it's easier to handle.  
\minisec{07.11.2011}
It should be possible to get this work further as a Bachelor Work.\\
We had brought up some ideas to build streets and cities dynamic with little pats of street tales. \\
We also could create a City (or a map) by our own. It is only good to use google street view or google maps if there is less work to do if we use it.\\
Rudy told us some scenarios which Rudy like to have for her studies. We will try to create the most of them but we have decided that if there are some elements in it which are animated or have to be controlled from outside, for example another car which we control the speed of, we would like to displace it to the Bachelor Work.  \\
The scenarios we like to build are the one with the tunnels. If we managed to do a car inside the tunnel which Rudy can control the speed of, we have to see if we have time. 
\minisec{14.11.2011}
Today we agree on a set of features that has to be included in the software we will deliver as the result of our project. Namely, these are:
\begin{itemize}
\item Loggin of the timestamp in the VidePlayer application
\item Integration of a cockpit view with speedometer
\item Different lighting of tunnels
\item Small city map
\end{itemize}
After having completed these tasks we will focus on updating the documentation and making a deliverable version of the software.\\
Tasks to be included in the bachelor thesis will be discussed in the meeting of the 28th November.
\minisec{21.11.2011}
We had different ideas, we would like to implement in the bachelor thesis. Important about these ideas is a relevance for the experiment they do with the driving simulator. \\
Rudy gave us also a lot we could implement in a further work. These are things like a bigger scene simulate a area in Zurich around the airport including the Bubenholz tunnel.  
\minisec{05.12.2011}
A simply aim of the bachelor thesis were set up and we named different possible tasks for it. \\
We discussed the structure we had set up for the PA documentation and get some good hints.
\minisec{12.12.2011}
We verified the definitive structure of the PA documentation. We settled the closing date for the documentation on the 23.12.2011. We are on the right was and absolutely in time with our work.