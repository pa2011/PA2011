\section{Journal}
\subsection*{26.09.2011}
The driving simulator with steering wheel and pedals is connected to the computer. There is a LabVIEW interface which reads the input from the devices. In a first step, Rudy would like to have a driving controlled only by the pedals. --> modified frame rate \\
After this she'd like to have a driving simulator with focus on tunnel entrances an exits.  
\subsection*{03.10.2011}
The driving simulator may be a fantasy environment or it could be a copy of a real environment. \\
Google Street View is no possibility to generate a 3D world because there are too many distraction in it. For example the other cars, people and traffic jam. The street view does not clearly distinguish between street and surfaces of other objects. Also sometimes there are only pictures available on the wrong side of the street. Another argument not to use Google Street View is that there are too larg distances between the single pictures, so the rendering is not fluent.\\
We would like to use the information from Google Earth to build a city like Zurich. We could use the street location information to build our own streets and then try to render already existing 3D models from Google 3D Warehouse into our virtual world. \\
There is a difficulty about Google Earth when it comes to creating scenes that take place in a tunnel since there is no height information available. A Possibility could be to implement it manually or to ignore spots which cause such problems. If we implement it manually we should define a fixed area where no problems occur. \\
We decided to use UDP sockets to extract data out of LabVIEW into our program and an external video player application to control the video.\\
In a first step we use LabVIEW to control an external application which plays a video with a configurable framerate. 
\subsection*{10.10.2011}
The journal and the timetabele have been set up and seem to be okay. There has to be an English version of the timetable so Rudy can follow the progress made in the project. \\
We have agreed on creating our own 3D world and extend it with some buildings from Google 3D Warehouse. There are already tons of finished 3D models of different buildings. We have to build the streets by ourselves because the streets in Google Earth are not as good as we need them. We will also create a tunnel in our own 3D world, which is impossible when rendering a scene with material of Google Street View. \\
We have presented the video we controlled with LabVIEW and the OGRE framework we would like to use. 
\subsection*{17.10.2011}
We have to calculate the delay time of the user interaction. A timestamp would be very helpful to study the delays. That's important for Rudy's further studies. \\
The video is now controlled by the pedals, and played in MPlayer. There is a batch file to start the application with different videos.
We switched our repository to github because there it's easier to maintain.  
\subsection*{07.11.2011}
It should be possible to get this work further as our Bachelor thesis.\\
We have brought up some ideas to build streets and cities dynamically with little pieces of street tales. \\
We also could create a city (or at least a map) by ourselves. It is only useful to use Google Street View or Google Maps if it bring a remarkable time advantage over doing the scenes manually.\\
Rudy told us some scenarios which she'd like to have for her studies. We will try to create the most of them but we have decided that if there are some elements in it which are animated or have to be controlled from outside, for example another car, we will move it to the Bachelor thesis.  \\
We agreed on building a tunnel scene but won't be able to create a controllable second car moving through that scene. 
\subsection*{14.11.2011}
Today we agree on a set of features that have to be included in the software we will deliver as the result of our project. Namely, these are:
\begin{itemize}
\item Logging of the timestamp in the VidePlayer application
\item Integration of a cockpit view with speedometer
\item Small city map
\end{itemize}
After having completed these tasks we will focus on updating the documentation and making a deliverable version of the software.\\
Tasks to be included in the Bachelor thesis will be discussed in the meeting of the 28th November.
\subsection*{21.11.2011}
We have got different ideas, we would like to implement in the Bachelor thesis. Important about these ideas is a relevance for the experiment they do with the driving simulator. \\
Rudy also gave us a lot of points we could implement in a further work. These are things like a bigger scene simulating an area in Zurich around the airport including the Bubenholz tunnel.  
\subsection*{05.12.2011}
A simple goal of the Bachelor thesis has been set up and we named different possible tasks for it. \\
We discussed the structure we had set up for the PA documentation and got some good hints from Mr. Früeh and Mr. Schlup. There has been a short interview with Prof. Menozzi on a Swiss television programme about his studies. Our driving simulator has been mentionned and shown on TV\footnote{http://www.videoportal.sf.tv/video?id=9410c439-0c70-4ee2-895c-f9339ec2809d}.
\subsection*{12.12.2011}
We verified the definitive structure of the PA documentation. We settled the closing date for the documentation on the 23.12.2011. We are on the right way and absolutely on time with our work.
