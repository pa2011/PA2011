\section{Einleitung}
\subsection{Ausgangslage}

Im Bereich der Fahrsimulatoren gibt es eine Vielzahl von verschiedenen Lösungen. Einige davon bestehen aus Filmmaterial, das abgespielt wird und der Fahrer muss auf die Bremse drücken, sobald ein bestimmtes Ereignis eintritt. Andere  bringen bereits eine virtuelle Welt mit, in der man sich mehr oder weniger frei bewegen bzw. frei fahren kann. Jedoch sind bei den meisten dieser Fahrsimulatoren feste Szenarien implementiert, die  nicht geändert werden können. Die Möglichkeiten eines Fahrsimulators werden hauptsächlich durch die Leistungsfähigkeit des Rechners, auf dem er läuft, limitiert. Eine besondere Herausforderung ist die Simulation von interaktivem Verkehr auf den Strassen. \\
Dieses Projekt wird in Zusammenarbeit mit der ETH Zürich durchgeführt. Im Rahmen unterschiedlicher Forschungsthemen sollen Probanden im Fahrsimulator verschiedene Strassensituationen antreffen. Eine dieser Forschungen bezieht sich auf die Auswirkung von Medikamenten im Strassenverkehr. Hierbei soll die Aufmerksamkeit und Reaktionsfähigkeit des Probanden vor und nach der Einnahme von Medikamenten getestet werden. Eine andere Studie untersucht die Augenreaktion beim Einfahren in Tunnels. Wenn der Portalbereich eines Tunnels von der Sonne beschienen wird, ist dieser sehr hell, der Innenbereich hingegen ist dunkel. Der Autofahrer ist also einem starken Helligkeitsunterschied ausgesetzt und die Augen müssen sich auf die neue Situation einstellen. Um herauszufinden, wie diese Anpassungsphase die Wahrnehmung des Autofahrers beeinflusst, werden solche Versuche im Simulator durchgeführt.\\
Die Hardwarekomponenten des Fahrsimulators bestehen aus einem Rechner, einem Projektor, einer speziell bemalten Wand als Projektionsfläche und einem Cockpit mit Autositz und Sicherheitsgurt. Im Cockpit befinden sich Gas- und Brems- und Kupplungspedale, ein Steuerrad mit verschiedenen Knöpfen und ein Schaltknüppel. Auf dem Rechner läuft \gls{labview}, das als Schnittstelle zwischen Eingabehardware und Simulationssoftware dient.

\subsection{Aufgabenstellung}
\subsubsection{Formulierung}
Um die ETH bei ihren Forschungen zu unterstützen wird im Rahmen dieser Projetkarbeit ein Fahrsimulator für die bestehende Hardwareumgebung entwickelt. Durch Eingaben im Cockpit soll der Proband das Fahrzeug durch eine 3D-Umgebung bewegen können und dabei eine möglichst realistisch Illusion des Autofahrens vermitteln bekommen. Die Szenen sollen unter anderem durch frei verfügbare Modelle aus \gls{google-3d-warehouse} aufgebaut werden. Alle Betriebszustände und Benutzereingaben registriert und aufgezeichnet werden um eine genaue Auswertung zu ermöglichen. 

\subsubsection{Aufteilung der Arbeit}
In einem ersten Schritt wird die Schnittstelle zwischen dem Cockpit und dem zu realisierenden Programm über LabVIEW implementiert. Zur Visualisierung wird ein Video-Player entwickelt, der bei Betätigung des Gas- und Bremspedals die Abspielgeschwindigkeit des Videos anpasst. Beim Videoplayer soll zudem die Speicherung von Daten in Log-Files  implementiert werden. Danach wird mit der erstellten Schnittstelle der Fahrsimulator realisiert. Ist dieser implementiert und getestet, werden verschiedene Szenen erstellt.

\subsection{Grober Zeitplan}
Die ersten zwei Wochen werden dazu vewendet, die Simulationsumgebung kennenzulernen. Danach werden weitere vier Wochen eingeplant um die Schnittstelle zwischen dem Cockpit und dem zu realisierendem Programm zu implementieren und sie mit einem Videoplayer zu testen. Für den Fahrsimulator und die zu erstellenden Szenen werden ca. vier Wochen eingeplant. Die letzten vier Wochen werden für die Dokumentation und eventuelle Ergänzungen benötigt. Ein detaillierter Zeitplan ist im Anhang \ref{detaillierter_zeitplan} zu finden.