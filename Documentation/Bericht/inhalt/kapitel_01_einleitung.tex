\section{Einleitung}
\subsection{Ausgangslage}

Im Gebiet der Fahrsimulatoren gibt es bereits eine Vielzahl von verschiedenen Lösungen. Einige davon bestehen aus Filmmaterial, das abgespielt wird und der Fahrer muss auf die Bremse drücken sobald ein bestimmtes Ereigniss eintritt. Andere Fahrsimulationen bringen bereits eine virtuelle Welt mit, in der man sich mehr oder weniger frei bewegen bzw. frei fahren kann. Jedoch sind bei den meisten von diesen Fahrsimulatoren bereits feste Szenarien implementiert die  nicht geändert werden können. Die Grenzen eines Fahrsimulators liegen vor allem in der Leistungsfähigkeit des Rechners auf dem die Simulation installiert werden soll. Zusätzliche Schwierigkeiten für Fahrsimulatoren ist die Simulation von Verkehr auf den Strassen. \\
Das Projekt wird in Zusammenarbeit mit der ETH Zürich durchgeführt. Im Rahmen verschiedener Forschungen sollen Probanden im Fahrsimulator verschiedene Strassensituationen antreffen. Eine dieser Forschungen bezieht sich auf die Einwirkung von Medikamenten im Strassenverkeht. Hierbei soll die Aufmerksamkeit und Reaktionsfähigkeit des Probanden vor und nach der Einnahme von Medikamenten getestet werden. Eine weitere Forschung untersucht das Einfahren in Tunnels mit einem Fahrzeug. Wenn der Portalbereich eines Tunnels von der Sonne beschienen wird ist dieser sehr hell, der innenbereich des Tunnels hingegen ist dunkel. Der Autofahrer ist also einem starken Helligkeitunterschied ausgesetzt und die Augen müssen sich auf die neue Situation einstellen. Um herauszufinden, wie sich diese Anpassungsphase der Augen auf die Wahrnemungsfähigkeit und Reaktionfähigkeit auswirkt, werden solche Versuche im Simulator durchgeführt. \\
Die Hardwarekomponenten des Fahrsimulators bestehen aus einem Rechner, einem Beamer, einre speziell bemalten Wand die als Projektionsfläche dient und einem Cockpit mit Autositz und Sicherheitsgurt. Im Cockpit befinden sich Gas- und Bremspedas sowie eine Kupplung, ein Steuerrad mit verschiedenen Knöpfen und eine Gangschaltung. Auf dem Rechner befindet sich eine LabVIEW-Umgebung. Es existiert bereits ein LabVIEW-Programm, das eine Anbindung an das Cockpit hat.

\subsection{Aufgabenstellung}
\subsubsection{Formulierung}
Um die ETH bei ihren Forschungen zu unterstützen wird im Rahmen dieser Projetkarbeit ein Fahrsimulator für die bestehende Simulationsumgebung entwickelt. Durch Manipulationen im Cockpit soll der Proband das Fahrzeug durch die 3D-Umgebung bewegen können. Die Fahrsimulation sollte dem Probanden die Illusion des Autofahren möglichts realistisch vermitteln. Durch den Einsatz von Google 3D Warehouse wird diese unterstützt. Zudem sollen alle Betriebszustände und Benutzereingaben registriert und aufgezeichnet werden um eine genaue Auswertung zu ermöglichen. 

\subsubsection{Aufteilung der Arbeit}
In einem ersten Schritt wird die Schnittstelle zwischen dem Cockpit und dem zu realisierenden Programm über LabVIEW implementiert. Zu diesem Zweck wird ein Video-Player entwickelt, der bei Betätigung des Gas- und Bremspedals die Abspielgeschwindigkeit des Video manipuliert. Beim Video-Player soll zudem bereits die Verwendung von Log-Files  implementiert werden. Danach wird mit der gleichen Schnittstelle der Fahrsimulator realisiert. Ist dieser implementiert und getestet, werden verschiedene Szenen für verschiedene Szenarien erstellt. 

\subsection{Grober Zeitplan}
Die ersten zwei Wochen werden dazu vewendet die Simulationsumgebung und dessen Anbindung kennenzulernen. Danach werden weiter vier Wochen eingeplant um die Schnittstelle zwischen dem Cockpit und dem zu realisierendem Programm zu implementieren und sie mit einem Videplayer zu testen. Für den Fahrsimulator und die zu erstellenden Szenen werden ca. 4 Wochen eingeplant. Die letzten 4 Wochen werden für das abschliessen der Dokumentation und eventuell verschiedenster kleinerer Arbeiten die noch ausstehen benötigt. Der detailierte Zeitplan ist im Anhang F zu finden. 