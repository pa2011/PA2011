\newglossaryentry{opengl}{
	name={OpenGL},
	description={Opensource Grafikbibliothek}
}
\newglossaryentry{directx}{
	name={DirectX},
	description={Microsofts Grafikbibliothek und Konkurrenz zu OpenGL}
}
\newglossaryentry{ogre}{
	name={OGRE},
	description={Object-Oriented Graphics Rendering Engine}
}
\newglossaryentry{mit-lizenz}{
	name={MIT-Lizenz},
	description={Eine vom Massachusetts Institute of Technology entworfene Lizenz zur Benutzung von Software. (Quelle: http://en.wikipedia.org/wiki/MIT\_License, Abruf: 03.12.2011)}
}
\newglossaryentry{shader}{
	name={Shader},plural={Shaders},
	description={Programm, welches auf der GPU läuft und der Berechnung von Licht-, Schatten- und weiteren Effekten von Materialien dient}
}
\newglossaryentry{szenengraph}{
	name={Szenengraph},
	description={Datenstruktur zur hierarchischen Organisation von von Objekten}
}
\newglossaryentry{spline}{
	name={Spline}, plural={Splines},
	description={Eine Linie die durch Interpolation von einzelnen Punkten definiert ist}
}
\newglossaryentry{seamless}{
	name={seamless},
	description={Nahtlos}
}
\newglossaryentry{node}{
	name={Node},
	description={Element des Szenengraphen, das eine Transformation und Modelle bzw. weitere Nodes enthält}
}
\newglossaryentry{vsync}{
	name={Vsync},
	description={Vertical Sync - bedeutet, dass mit dem Neuberechnen eines Bildes gewartet wird, bis der gesamte Bildschirminhalt gezeichnet wurde. Dadurch wird der Verzerrung des Bildes durch schnelle Bewegungen vorgebogen.}
}
\newglossaryentry{callback-methode}{
	name={Callback-Methode},
	description={Methode, die von einem externen Programmteil aufgerufen wird}
}
\newglossaryentry{ogre-xml}{
	name={OGRE-XML},
	description={Von der OGRE-Community entworfenes Format zur Speicherung von 3D-Modellen. Es existieren Plugins für diverse 3D-Modellierungstools, mit denen Objekte in diesem Format gespeichert werden können}
}
\newglossaryentry{gui}{
	name={GUI},
	description={Abkürzung für Graphical User Interface (zu deutsch: Grafische Benutzeroberfläche)}
}
\newglossaryentry{untersteuern}{
	name={Untersteuern},
	description={Rutschen des Fahrzeugs über die Vorderachse in der Kurve}
}
\newglossaryentry{1st-person}{
	name={1st-Person},
	description={Ich-Perspektive}
}
\newglossaryentry{3rd-person}{
	name={3rd-Person},
	description={Kameraeinstellung, bei der die Kamera aus dem Blickwinkel einer dritten Person (Beobachter) filmt}
}
\newglossaryentry{google-3d-warehouse}{
	name={Google 3D Warehouse},
	description={Eine Plattform, auf der Modelle aus Google SketchUp veröffentlicht und kostenlos heruntergeladen werden können. http://sketchup.google.com/3dwarehouse}
}
\newglossaryentry{google-sketchup}{
	name={Google SketchUp},
	description={Software zum Erstellen von Objekten. Sie wurde ursprünglich zur Modellierung von Gebäuden für Google Earth entwickelt, unterstützt aber auch andere bekannte 3D-Formate}
}
\newglossaryentry{cinema4d}{
	name={Cinema4D},
	description={Kommerzielles 3D-Modellierungstool vom Softwarehersteller MAXON}
}
\newglossaryentry{a-saeule}{
	name={A-Säule},
	description={Die vordersten beiden Säulen beim Auto, die das Dach tragen}
}









