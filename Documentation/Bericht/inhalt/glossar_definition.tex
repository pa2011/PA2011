\newglossaryentry{opengl}{
	name={OpenGL},
	description={Opensource Grafikbibliothek}
}
\newglossaryentry{directx}{
	name={DirectX},
	description={Microsofts Grafikbibliothek und Konkurrenz zu OpenGL}
}
\newglossaryentry{ogre}{
	name={OGRE},
	description={Object-Oriented Graphics Rendering Engine}
}
\newglossaryentry{mit-lizenz}{
	name={MIT-Lizenz},
	description={Eine vom Massachusetts Institute of Technology entworfene Lizenz zur Benutzung von Software. (Quelle: http://en.wikipedia.org/wiki/MIT\_License, Abruf: 03.12.2011)}
}
\newglossaryentry{shader}{
	name={Shader},
	description={Programm, welches auf der GPU läuft und der Berechnung von Licht-, Schatten- und weiteren Effekten von Materialien dient}
}
\newglossaryentry{szenengraph}{
	name={Szenengraph},
	description={Datenstruktur zur hierarchischen Organisation von von Objekten}
}

\newglossaryentry{spline}{
	name={Spline}, plural={Splines},
	description={Eine Linie die durch Interpolation von einzelnen Punkten definiert ist}
}
\newglossaryentry{seamless}{
	name={seamless},
	description={Nahtlos}
}
\newglossaryentry{node}{
	name={Node},
	description={Element des Szenengraphen, das eine Transformation und Modelle bzw. weitere Nodes enthält}
}
\newglossaryentry{vsync}{
	name={Vsync},
	description={Vertical Sync - bedeutet, dass mit dem Neuberechnen eines Bildes gewartet wird, bis der gesamte Bildschirminhalt gezeichnet wurde. Dadurch wird der Verzerrung des Bildes durch schnelle Bewegungen vorgebogen.}
}
\newglossaryentry{callback-methode}{
	name={Callback-Methode},
	description={Methode, die von einem externen Programmteil aufgerufen wird}
}
\newglossaryentry{ogre-xml}{
	name={OGRE-XML},
	description={Von der OGRE-Community entworfenes Format zur Speicherung von 3D-Modellen. Es existieren Plugins für diverse 3D-Modellierungstools, mit denen Objekte in diesem Format gespeichert werden können}
}