\section{Das OGRE-Framework}

\subsection{Was ist das OGRE-Framework}

\begin{wrapfigure}{r}{0.4\linewidth}
	\includegraphics[width=1\linewidth]{src/OgreLogo.png}
	\caption{OGRE Logo} % Titel der Grafik
	\label{OGRE Logo} % Labelname
\end{wrapfigure}

OGRE (Object-Oriented Graphics Rendering Engine) ist eine Grafikengine zur Echtzeitdarstellung von dreidimensionalen Szenen. Sie ist in C++ geschrieben und ihre Verwendung unterliegt der MIT-Lizenz. Durch den modularen Aufbau und die Unterstützung auf verschiedenen Plattformen erweist sich OGRE als sehr flexibel und mächtig. OGRE selbst benutzt die Grafikbibliotheken \gls{opengl} und \gls{directx} zum hardwarebeschleunigten Rendern der Szenen und setzt automatisierte Optimierungsalgorithmen zur Geschwindigkeitsgewinnung ein. Mittlerweile existiert eine grosse und aktive Community, welche das OGRE-Framework ständig weiter entwickelt und verbessert.

\subsection{Weshalb setzen wir OGRE ein?}

Im Vergleich zur direkten Verwendung von \gls{opengl} oder \gls{directx} bringt der Einsatz einer Grafikengine wie OGRE eine Vielzahl von Vorteilen mit sich:

\minisec{Abstraktion} \gls{opengl} und \gls{directx} sind zwei komplett verschiedene Bibliotheken. Hat man sich für eine von beiden entschieden, so bedeutet das Umsteigen auf eine andere unter Umständen das Neuschreiben des kompletten Codes. Ogre abstrahiert die Verwendung dieser Grafikbibliotheken und ermöglicht den selben Code entweder mit OpenGL oder DirectX laufen zu lassen.

\minisec{Geschwindigkeit} text
