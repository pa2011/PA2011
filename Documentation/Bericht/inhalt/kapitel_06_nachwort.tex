\section{Nachwort}
\subsection{Rückblick}
Die Projektarbeit Fahrimulator konnte erfolgreich und termingerecht abgeschlossen werden. Sämtliche Anforderungen der Aufgabenstellung wurden erfüllt. Das Fahrzeug kann durch Eingaben im Cockpit navigiert werden und es stehen zwei unterschiedliche Szenen zur Verfügung. Das System läuft weitgehend robust.\\
Mehr Zeit als zu Anfang geplant, benötigten wir vorallem für die Erstellung der verschiedene Szenen. Obwohl wir mit dem Cinema 4D ein sehr gutes Tool zu Hand hatten, mussten Matrialien und Texturen manuell erstellt weden. Dies führte zu Fehlern die wiederum viel Zeit benötigten, um behoben zu werden.\\
Rückblickend sind wir froh, dass wir uns zu Beginn gegen einen Einsatz von google maps und google Street View entschiden haben. Durch die eingene modelierung der 3D Umgebungen konnten so gute Resultate erziehlt werden, wie sie mit den beiden anderen Tools niemals möglich gewesen wären. 
\subsection{Danksagung}
An dieser Stelle möchten wir unseren beiden Projektbetreuern  Prof. Dr. Peter Früh und Prof. Martin Schlup danken für ihre Betreuung, Unterstüzung und Anregungen während der gesamten Projektarbeit. Ebenso möchten wir dem Team an der ETH Zürich, Ying-Yin Huang und Prof. Dr. Marino Menozzi für eine erfolgreiche und angenehme Zusammenarbeit danken. Des weiteren danken wir Daniela Zehnder und Katrin Achtnich für die Überprüfung der Rechtschreibung und Grammatik. 
