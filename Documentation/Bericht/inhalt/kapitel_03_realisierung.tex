\section{Vorgehen und Methoden}
Wie bereits in der Systembeschreibung beschrieben, wird der Fahrsimulator in dei Hauptkomponenten unterteilt. Zum Fahrsimulator gehört ein UDP-Listener, ein Szenen-Manager und das Hauptprogramm. Da die Verbindung zwischen dem Cockpit und dem Fahrsimulator mit einer Netzwerkschnittstelle realisiert wird, ist es möglich die Anbindung an das Cockpit und den Fahrsimulator physisch auf zwei Rechnern zu betreiben. Eine Alternative ist die Umsetzung mit dem Ogre Framwork (Siehe Anhang C). Dieses Framework bietet umfassende Lösungen um Steuerradär und Pedalen, wie sie in userem Cockpit vorhanden sind, anzusteuern.\\
Da die 3D-Umgebung mit fortscheitendem Projekt ausgebaut wird, könnte sich die notwendige Rechenleistung erheblich steigern. Folgen ungenügender Leistung sind unregelmässige Bewegungen (Lags) in der virtuellen Umgebung oder sogar bis zum Absturz des gesamten Programms führen. Dieses Problem hat zur Folge, dass der Fahrsimulator einer hohen Portierbarkeitanforderung genügen muss. \\
Aus diesem Grund ist die erste Variante mit der Netzwerkschnittstelle für das Fahrsimulatorprojekt besser geeignet. Die andere Variante wird verworfen.
\subsection{LabView modifizierung}
Eine LabVIEW-Umgebung existiert bereits auf dem Rechner an dem das Cockpit angeschlossen ist. Dazu existierte ein LabVIEW-Programm, dass bereits die Eingaben im Cockpit einliest. Da dieses Programm jedoch nicht genau den Anforderungen entspricht, wird ein neues LabVIEW-Programm realisiert. Dieses Programm wird so erweitert das diese Eingaben in Form von definierten Parameter als UDP-Packet permanent gesendet werden. Zusätzlich wird ein zweiter UDP-Socket eingerichtet um Packete die vom Fahrsimulator gesendet werden zu empfangen. Die Daten der empfangenen Packete werden von LabVIEW in ein Log-File gespeichert. 
% ev Bild
\subsection{UDP-Socket}
Die Paramter werden vom LabVIEW-Programm permanent gelesen und mindestens 50 mal pro Sekunde wird ein UDP-Packet gesendet. Um die Zeit zwischen der Eingabe des Probanden und derReaktion des Fahrzeuges im Simulator möglichst gering zu halten, müssen ankommende Packete sofort gelesen werden. Angesitchts der hohen Anzahl an Datenpacketen pro Zeitraum, ist es vernachlässigbar, wenn einige Packete verloren gehen oder vertauscht werden. Das UDP-Protokoll gewährleistet genau diese Eigenschaften. \\

\begin{figure}[H]
\centering 
\includegraphics[scale=0.5]{src/koordinatensystem.png}
\caption{Koordinatensystem} % Titel der Grafik
\label{Koordinatensystem} % Labelname
\end{figure}

Die von LabVIEW empfangenen Daten sind in erster Linie Parameter die den Zustand der Pedalen und des Steuerrades representieren. Die beiden Pedalen werden in einem Koordinatensystem wie in Abbildung \ref{Koordinatensystem} auf der y-Achse abgebildet. Die positive y-Richtung quantifiziert das Gas und die negative y-Richtung das Bremspedal. Die Intensität beider Pedalen wird durch 32767 bzw. 32768 ganzzahlige Werte identifiziert. Wenn also keines der beiden Pedalen gedrückt ist, wird dies durch den y-Wert 0 dargestellt. Ein voll gedrücktes Gaspedal entspricht dem y-Wert 32767 und dementsprechend ein voll gedrücktes Bremspedal dem Wert -32768. Der x-Wert im Koordinatensystem quantifiziert den Einschlagswinkel des Steuerrades. Ist das Steuerrad in der neutralen Stellung, entspricht dies dem x-Wert 0. Wenn das Steuerrad vollständig nach rechts eingeschlagen ist, wird dies durch den positiven x-Wert 32767 identifiziert. Umgekehrt wird ein vollständiger Einschlag nach links durch den negativen x-Wert -32768 identifiziert. Es können zusätzlich weitere Parameter wie Tastendrücke am Steuerrad, übermittelt werden, wenn dies erforderlich sein sollte. Zusätzlich zu den Parametern des Cockpits wird von LabVIEW auch noch ein Timestamp gesendet. Der Timestamp wird von UDP-Listener empfangen und nach dem Abspeichern der Parameter wieder zurückgeschickt. Somit kann festgestellt werden, wieviel Verzögerung die Netzwerkschnittstelle verursacht. \\
Da das LabVIEW-Programm und der Simulator nicht synchronisiert sind, wird der DP-Listener zur Überbrückung verwendet. Der UDP-Listener speichert alle Parameter in Variabeln ab. Diese Variabeln werden vom Hauptprogramm ausgelesen und interpretiert. Dieses auslesen der Variabeln erfolgt nun synchron, da das Hauptprogramm nur dann Daten vom UDP-Listener liest, wenn dieses dazu bereit ist.

\subsection{Virtual Reality}
Für den Fahrsimulator wird eine virtuelle dreidimensionale Umgebung benötigt, in der man sich bewegen kann. Diese Umgebung, auch Szene genannt, setzt sich aus verschiedenen Objekten zusammen. Die Struktur eines solchen Objektes wird duch ein Mesh-File beschrieben. In einer Szene kann das gleiche Objekt mehrmals an verschiedenen Positionen vorkommen. Das Objekt kann durch eine skalierung auch einmal gross und einmal klein erscheinen. Oder man kann sie durch eine Rotation unterschiedlich machen. Zusätzlich wird mindestens ein Material-File benötigt in dem sämtliche Texturen und Materialien, die von Objekten der Szene verwendet werden, definiert sind.
Ein Beispiel von einem Matrial wie es ein einem material-File definiert ist wird im Listing \ref{example_materialfile_material} illustriert.
\begin{lstlisting}[caption={Beispiel aus dem material-File für ein Material},label={example_materialfile_material}]
material city_ground
{
	technique
	{
		pass
		{
			diffuse 0.0 0.0 0.0 1.00
			ambient 0.75 0.75 0.75 1.00	
		}
	}
}
\end{lstlisting}
Nach dem Keyword material wird der Name des Materials angegeben. Als technique-Block wird hier einer angegeben. Es können mehrere Blöcke angegeben. Diese Blöcke können optional auch Namen gegeben werden um zu identifizieren wofür die einzelnen Blöcke sind. Ausgeführt wird jedoch immer nur einer der Blöcke wobei der erste Block der bevorzugteste ist und der unterste Block die letzte Fallbacklösung wäre. Im pass-Block werden die Reflektionswerte und die Farbe angegeben. Es können mehrer pass-Blöcke angegeben werden die jeweils hintereinander ausgeführt werden und sich überlagern. Die Zahlen repräsentieren die RGB-Werte und können zwischen 0 und 1 liegen. Die vierte Zahl representiert den Alphawert der Farbe. Der Alphawert quantifiziert die Sichtbarkeit des Materials wobei 1.0 voll sichtbar und 0.0 unsichtbar ist. \\

Weiter zeigt das Listing \ref{example_materialfile_texture} die definition einer Textur in einem material-File.
\begin{lstlisting}[caption={Beispiel aus dem material-File für eine Textur},label={example_materialfile_texture}]
material city_street_h
{
	technique
	{
		pass
		{
			diffuse 0.80 0.80 0.80 1.00
			
			texture_unit
			{
				texture street_h.png 2d 4
				filtering anisotropic
			}
		}
	}
}

\end{lstlisting}

\subsection{Hautpprogramm}
\lstinputlisting{listings/Karte.java}


%Sequenzdiagramm
%Blockdiagramm

