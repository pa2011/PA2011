%
% Headerdatei der Projektarbeit
%

\documentclass[
		a4paper,
		12pt,
		oneside,
		%openright,
		parskip,
		chapterprefix,%       Kapitel anschreiben als Kapitel
]{scrreprt}

\usepackage{moreverb}

%Deutsche Trennungen, Anführungsstriche und mehr:
\usepackage{german, ngerman}
\usepackage[german]{babel}

%Eingabe von ü,ä,ö,ß erlauben
\usepackage[utf8]{inputenc}

%Zum Einbinden von Grafiken
\usepackage{graphics}

%Ein Paket, das die Darstellung von "Text, wie er eingegeben wird"
%erlaubt: Also
%\begin{verbatim} \end{document}\end{verbatim} erzeugt die Ausgabe von
%\end{document} im Typewrites-Style und beendet nicht das Dokument.
\usepackage{verbatim}

%Source-Code printer for LaTeX
\usepackage{listings}

%Darstellung des Glossars einstellen
\usepackage[
		style=super, 
		header=none, 
		border=none, 
		number=none, 
		cols=2,
		toc=true
]{glossary}

\makeglossary

% Bereitstellung Hyperlinkfunktionen (PDF) (muss als letztes Paket geladen werden)
\usepackage[
	colorlinks=true,
	linkcolor=black,
	citecolor=black,
	pdfpagelabels,
	pdftitle={Projektarbeit},
	pdfsubject={Software für Personeneinsatzplanung},
	pdfkeywords={},
	pdfauthor={Michael Schwarz (schwami0), Andreas Ruckstuhl (rucksand)}
]{hyperref}
