\section{Ausgangslage} 

Um einen gegeigneten Lösungsansatz zu finden, ist es wichtig, zu erst festzustellen, wie Dispositionz zur Zeit erledigt wird. Dazu werden die folgenden Punkte analysiert:

\begin{itemize}
	\item Welche Informationen und Struktur enthält eine erstellte Disposition. % Wer, wann, was, wo. Wie genau sind diese Informationen spezifiziert.
	\item Welche Werkzeuge und Konventionen werden zum Erstellen und Kommunizieren einer Disposition verwendent. % Excel, Wand mit Blättern, Farben
	\item Wer ist an der Erstellung einer Disposition beteiligt. % z.B. die Garten-Abteilung wird sepparat Disponiert. Ein Teil der Disposition wird durch die Mitarbeiter übernommen (z.B. Reihenfolge der Liegenschaften, wann weniger häufige Aufgaben gemacht werden.)
	\item Welche Informationen stehen der Disponentin zur verfügung die einen Einfluss auf die Disposition haben könnten. % z.B. dass immer ein Mitarbeiter eine Liegenschaft schon kennen sollte.
	\begin{itemize}
		\item Wie werden diese Informationen bezogen.
		\item Wie weit im Voraus sind diese Informationen verfügbar.
		\item Wie sind diese Informationen strukturiert. % z.B. sind es Ja-Nein fragen, Mengen, Bewilligungen die Mitarbeiter haben im Gegensatz zu einer wagen Wetterprognose oder einer Einschätzung der Fähigkeiten eines Mitarbeiter durch die Disponentin
	\end{itemize}
	\item Welches sind die Einschränkungen die die Disponentin in Betracht ziehen muss.
	\item Wie weit im Voraus wird eine Disposition erstell und wie häufig und in wie fern wird diese danach noch geändert.
	\item Welche
\end{itemize}

\subsection{Informationsgehalt einer Disposition}

% Den ersten Teil vielleicht in ein Glossar verschieben
Eine Disposition bezeichnet die Zuordnung einer bestimmten Tätigkeit einem bestimmten Mitarbeiter oder einem Team in einem bestimmten Zeitfenster. Da ein Grossteil der Aufgaben jede Woche woche erledigt werden müssen, wir die Disposition meist für eine Woche am Stück erstellt.

Dies ist aber nur eine Idealvorstellung, denn die zur Zeit angewendete Methode zur Erstellung einer Disposition macht gebrauch von vielen Konventionen, welche es überflüsig machen, alle Details einer Details einer Disposition auszuarbeiten. Deshalb muss die anfängliche Aussage relativiert werden:

% TODO: Sind die verwendeten Beispiele realistisch?
Der Vertrag, welcher mit dem Verwalter einer Liegenschaft abgeschlossen wird, enthält oft eine Grosszahl an Aufgaben, welche mit unterschiedlicher Häufigkeit erledigt werden müssen. Beispielsweise kann vereinbart werden, dass einmal in der Woche das Treppenhaus, jeden zweiten Monat die Fenster und einmal jährlich die Garage gereinigt wird.

Bei den meisten Zuordnungen werden den Mitarbeitern nicht bestimmten Aufgaben zugeordnen sondern lediglich das "Erledigen" einer bestimmten Liegenschaft. Es liegt dann im Ermessen der Mitarbeiter weniger häufige Arbeiten auf den Wochenrythmus aufzuteilen. Grössere, seltenere Aufgaben werden wiederum in die Planung mit einbezogen, da z.B. das reinigen aller Fenster einer grösseren Liegenschaft mehr Mitarbeiter als üblich mehrere Tage in Anspruch nehmen kann.

Mitarbeiter werden zur Disponierung meist in ein Team von zwei Mitabeitern gruppiert. Diese beiden Mitarbieter erledigen dann über einen längener Zeitraum die ihnen zugeteilten Aufgaben zusammen.

Um den Aufwand bei der manuellen Planung drastisch zu verringern wird werden die meisten zu erledigen Aufgaben oder Liegenschaften in so genanten "Touren" gruppiert. Ein Tour entspricht mehr oder weniger den Aufwand der in einem Vormittag oder Nachmittag durch ein Zweier-Team erledigt werden kann. Eine Tour wird nur aufgebrochen und neu zusammengesetzt, wenn z.B. Liegenschaften hinzukommen oder wegfallen oder sich der Aufwand einer Liegenschaft stark ändert (z.B. durch übernehmen von zusätzlichen Arbeiten).


\subsection{Werkzeuge und Konventionen}

Eine Disposition wird fast ausschliesslich mit dem Tabellenkalkulations-Programm Microsoft Excel erstellt. Dabei wird für jeden Tag einer Woche eine Spalte und für jeden Mitarbeiter eine Zeile verwendet. Mitglieder eines Teams sind in benachbarten Zeilen gruppiert. Die beiden Zellen jeder Team-Tag-Kombination werden verwendet um für den Vormittag und Nachmittag die Tour zu bezeichnen oder direkt Liegenschaften und/oder Aufgaben aufzuzählen. In dieser Tabelle werden auch Ferien der Mitarbeiter eingetragen. Für jede Woche wird diese Tabelle neu erstellt, meist mit der vorangehenden Woche als Vorlage.

In einer zweiten Tabelle werden die Touren in ihre einzelnen Tätigkeiten aufgeschlüsselt. Den meisten Touren wird eine Farbe zugeordnet. Dies Farbe wird dann in der ersten Tabelle verwendet, um Touren einfach wiederzuerkennen. Dies ist aber keine Regel, da gewisse Farben auch verwendet werden um andere Details hervorzuheben, z.B. Ferien von Mitarbeitern oder der Belegung von grösseren, spezielleren Aufgaben. Gewissen Touren ist auch keine Farbe zugeordned, in Fällen in denen es dem Zweck nicht gedient hat. Diese Tabelle wird bei bedarf aktualisiert.

Nachdem die Planung für eine Woche gemacht wurde, wird die Tabelle gedruckt und an einer für alle Mitarbeiter leicht zugänglichen Stelle aufgehängt. Die Touren-Tabelle wird gedruck wenn Änderungen daran gemacht wurden und in der Nähe der ersten Tablle plaziert. Es werden immer mehrere Wochen im voraus geplant und aufgehängt. Diese gedruckten Versionen dienen oft dazu, kleine Änderungen mit Bleistift einzutragen und Probleme in der Planung daran zu diskutieren. Die meisten Mitarbeiter inormieren sich anhand dieser all-morgendlich daran, was sie an einem gegebenen Tag zu erledigen haben.


\subsection{Einschränkungen}
\begin{itemize}
	\item Mit den Mitarbeitern vereinbarte Arbeitszeiten und Ferien.
	\item Rechtliche Einschränkungen bezüglich Pausen, Maximal-Arbeitszeiten und Überzeit.
	\item Mitarbeiter können z.B. von der Gärtnerei "ausgeliehen" werden.
	\item Gewisse Arbeiten müssen angekündigt werden (z.B. wenn die Wasserzufuhr unterbrochen).
	\item Für gewisse Arbeiten müssen Maschienen gemietet werden (z.B. eine Hebebühne um Fenster von aussen zu reinigen.).
\end{itemize}




